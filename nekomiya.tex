\documentclass[a4paper, 11pt, titlepage]{jsarticle}
\usepackage[dvipdfmx]{graphicx}
\usepackage{listings}
\usepackage{amsmath}
\usepackage{url}
\usepackage{pgfgantt}%追加内容

\title{知能情報実験III(データマイニング班)\\発表タイトルやテーマ名}
\author{グループの学籍番号列挙 xxx, yyy, zzz}
\date{提出日:20xx年x月x日}
\begin{document}
\maketitle
\tableofcontents
\clearpage

\abstract{概要}
nothing else

\section{はじめに}%yukina
nothing else
\section{実験方法}%yukina

\subsection{実験目的}
%実験を通して明らかにしたいこと、確認したいこと、検証したいことを述べよう。
%データセットもモデルもサイトから運用してるので適当に内容は絡めておくべきか?
本実験を通して、指紋認証についての理解を深めると同時に指紋認証の正答率を向上させる改善手法を模索する。

\subsection{データセット構築}
%既にどこかで公開されているデータセットをダウンロードして利用したのならば、そのURLを掲載する程度で構いません。
%独自構築した場合にはその構築方法を述べよう。
データセットは公開されている以下のデータセットを用いた。\\
https://www.kaggle.com/ruizgara/socofing


\subsection{モデル選定}
%どのようなモデルやアルゴリズムを利用したのか、何故それを選んだのか述べよう。
アルゴリズムは以下のものを用いた。\\
https://www.kaggle.com/brianzz/subjectid-finger-cnnrecognizer

本アルゴリズムの運用において、指紋認証の正答率が十分に高い点、可読性に優れており改善手法を模索しやすい点などにおいて使用することにした。%なんかダメな気がするぅ


\subsection{パラメータ調整}
%手動調整が必要なパラメータについて、どのように調整したのか述べよう。
改善手法を模索するにあたって、既に調整されていたepochの回数などを変更し、正答率の向上に寄与している化を検証した。



\newpage
\section{実験結果}%uema


\section{考察}
nothing else

\section{意図していた実験計画との違い}
%グループワークとして2ヶ月程度の時間が用意されていた。
%ガントチャート\ref{ganttchart}等、何かしら工夫して全体の計画を述べよう。
%これらの期間をどのように使おうとし、実際どうだったのかについて自己評価(振り返り)してみよう。
%大きなズレがある場合それは何故起きたのか、どうやればそのギャップを縮められそうか検討してみよう。

%https://ctan.org/pkg/pgfgantt
当初予定していた実験計画は以下の通りとなる。
下記の予定は11/17日に作成されたもの。後々詳しく決められた締め切り予定には(追加内容:)として記述する。%ラベルじゃ無いけど。

%超絶めんどくさくなった。
\begin{table}[htb]
	\begin{tabular}{|l|c|c|}
		\hline
		講義時間(日次) & 実験計画 & 実際に行った内容	\\ \hline \hline
		6週目(11/17)
		 &\begin{tabular}{l}
		 	テーマ決め\\
			データセット探し\\
			などなど
		\end{tabular}
		 & \begin{tabular}{l}
		 	テーマ決定(画像認証/fingerprint)\\
			データセット探し\\
			画像認証に対しての学習
		\end{tabular}	\\ \hline
		7週目(11/24)
		 &\begin{tabular}{l}
		 	特徴ベクトルの抽出や\\
			コード探して理解したり
		\end{tabular}
		 &\begin{tabular}{l}
		 	指紋の特徴量の分析\\
			CNNの画像認証コードの検索
		\end{tabular}	\\ \hline
		8週目(12/1)
		 &画像認識をコードに落とし込む
		 &\begin{tabular}{l}
		 	参考するCNNコードの決定\\
			コードの内容の分析・実行
		\end{tabular}		\\ \hline
		9週目(12/8) 
		 &\begin{tabular}{l}
		 	実験開始する?\\
			出来上がったコードを動かす?
		\end{tabular}
		 &\begin{tabular}{l}
		 	コードの内容の分析\\
			実行2nd
		\end{tabular}	\\ \hline
		10週目(12/15)
		 & コードや特徴ベクトルの調整1
		 &\begin{tabular}{l}
		 	コードの内容の分析3rd\\
			amaneによる実行開始\\
			実験目的の草案作成
		\end{tabular}	\\ \hline
		11週目(12/22)
		 & (未決定)
		 &\begin{tabular}{l}
		 	コードの内容の分析4th・実行2nd\\
			amaneで画像データ作成は難しいと判断\\
			来年度に向けたの引継ぎ
		\end{tabular}	\\ \hline
		12週目(1/5)
		 & (未決定)
		 &\begin{tabular}{l}
		 	コードの内容の分析5th\\
			epoch数を変えた大規模実験\\
			Python/Keras環境の統一
		\end{tabular}	\\ \hline
		13週目(1/12)
		 & 改善実験?
		 & \begin{tabular}{l}
		 	コードの内容の分析6th\\
			batch\_sizeやLeakyReLUを変更した実験
		\end{tabular}	\\ \hline
		14週目(1/19)
		 & (未決定)
		 &\begin{tabular}{l}
		 	レポートの作成\\
			amaneによる実験 
		\end{tabular}	\\ \hline
		15週目(1/26)
		 &\begin{tabular}{l}
		 	レポート・プレゼン資料作成\\
			(追加内容:レポート初期版の提出日)
		\end{tabular}
		 &\begin{tabular}{l}
		 	レポート・プレゼン資料作成\\
			レポート初期版の作成
		\end{tabular}	\\ \hline
		(1/26) &  & レポート・プレゼン資料作成	\\ \hline
		期末日(2/2) & 最終発表 & 最終発表	\\ \hline
		(2/16) & (追加内容:Github公開) & 公開済み	\\ \hline
	\end{tabular}
\end{table}

実験計画を作成した段階では後半の進捗の進み具合の判断が困難だったため、前半の予定と締め切りに合わせた大まかな目標を設定し、他は未定とした。未定の欄を除いて概ね実験計画と実際の進捗に大きな乖離はなく、順調に進んだと判断できる。


%\usepackage{pgfgantt}    ←追加内容←追加しませんでした←結局追加しました。←面倒なので追加しません
%\begin{ganttchart}{1}{12} \gantttitle{2011}{12} \\ \gantttitlelist{1,...,12}{1} \\
%\ganttgroup{Group 1}{1}{7} \\
%\ganttbar{Task 1}{1}{2} \\
%\ganttlinkedbar{Task 2}{3}{7} \ganttnewline
%\ganttmilestone{Milestone}{7} \ganttnewline
%\ganttbar{Final Task}{8}{12}
%\ganttlink{elem2}{elem3}
%\ganttlink{elem3}{elem4}
%\end{ganttchart}


\section{まとめ}
nothing else


\end{document}
