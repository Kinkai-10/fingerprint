\documentclass[a4paper, 11pt, titlepage]{jsarticle}
\usepackage[dvipdfmx]{graphicx}
\usepackage{listings}
\usepackage{amsmath}
\usepackage{url}


\title{知能情報実験III(データマイニング班)\\発表タイトルやテーマ名}
\author{グループの学籍番号列挙 xxx, yyy, zzz}
\date{提出日:20xx年x月x日}
\begin{document}
\maketitle
\tableofcontents
\clearpage

\abstract{概要}
nothing else
\section{はじめに}
nothing else

\section{実験方法}
nothing else

\newpage
\section{実験結果}
%事実として得られた結果を示そう。
%なお、以下の点に留意すること。

実験結果として、以下の結果が得られた。%ここの条件(batch_sizeとかmodelの中身とか)の補足をお願いします。
%epoch10の値の共通項がコピペ事故起こしてそうですが大丈夫ですか?

\subsection{エポック数を1から30までに変化}
\begin{table}[htb]
  \begin{tabular}{|l|c|c|}
    \hline
    epoch & subjectID accracy & fingerNum accracy \\ \hline
    1 & 01.5333333052694798 \% & 63.01666498184204 \%  \\ \hline
    3 & 61.283332109451294 \% & 89.48333263397217 \% \\ \hline
    5 & 96.35000228881836 \% & 98.15000295639038 \%  \\ \hline
    10 & 99.71666932106018 \% & 99.71666932106018 \%  \\ \hline
    20 & 99.73333477973938 \% & 99.88333582878113 \%  \\ \hline
    30 & 99.73333477973938 \% & 99.90000128746033 \% \\ \hline
  \end{tabular}
\end{table}

\subsection{epochを20に固定し、batch\_sizeを変更}%アンダーバーは関数記号?エラーの元。ただの記号にするためのバックスラッシュが必要
\begin{table}[htb]
  \begin{tabular}{|l|c|c|}
    \hline
    batch\_size & subjectID accuracy & fingerNum accuracy  \\ \hline
    32 & 99.73333477973938 \% & 99.90000128746033 \% \\ \hline
    64 & 99.73333477973938 \% & 99.88333582878113 \% \\ \hline
    128 & 99.71666932106018 \% & 99.88333582878113 \% \\ \hline
  \end{tabular}
\end{table}

%ここの条件(batch_sizeとかepochとか)の補足をお願いします。
\subsection{活性化関数に変更}
\begin{table}[htb]
  \begin{tabular}{|l|c|c|}
    \hline
    活性化関数 & subjectID accracy & fingerNum accracy \\ \hline
    LeakyReLU (alpha=-0.5) & 99.43333268165588 \% & 99.88333582878113 \% \\ \hline
    LeakyReLU (alpha=0.3) & 99.71666932106018 \% & 99.73333477973938 \% \\ \hline
    LeakyReLU (alpha=0.5) & 99.6999979019165 \% & 99.6333360671997 \% \\ \hline
    sigmoid & 99.73333477973938 \% & 99.86666440963745 \% \\ \hline
    tanh & 99.73333477973938 \% & 99.88333582878113 \% \\ \hline
  \end{tabular}
\end{table}

\clearpage

\section{考察}
初めに、元のサンプルコードの条件はエポック数20のバッチサイズ64、活性化関数は中間層でReLU関数を用いて、出力層に対してsoftmax関数を用いている。結論から言うと、どの条件下においても大きな変化は見られなかった。精度を向上させることができた条件は、エポック数を20から30に増やすこと、バッチサイズを64から32に変更することの2つであった。また、両者とも精度が上がったのは fingerNum accracy、つまりどの指であるかの識別するものであった。この原因と、その他の条件において精度が上昇しなかった原因について考える。

初めに、fingerNum accracyのみ精度が向上したことについて考える。まず、エポック数とは学習する世代のことであり、これを多くしていくことで細かい識別が可能になる。直感的に考えれば、10本のうちどの指かを識別するよりも、600人のうち誰の指紋であるかを識別する方が難しいはずであるが、このことからfingerNum accracyの方が細かい識別が必要になると考える。

次に、活性化関数を変更した場合について考える。今回は画像の識別であるため、活性化関数も多くの値を表現できるものが良いと考えた。0と1だけで識別するのと、1から100までを用いて識別するのは後者の方がより細かいものを表現できるはずである。実際、元のサンプルコードは正の値はそのまま使用するReLU関数を用いていた。そこで、負の値も一定の割合で使用するLeakyReLU関数を用いたが結果は精度がやや落ちる程度だった。また、sigmoid関数とtanh関数も用いた。これらは入力された値を、0から1、-1から1の実数にする関数である。これまでの考えだと精度は下がると予測したが、LeakyReLU関数を用いた結果とあまり大差はなかった。これらのことを見ると活性化関数に精度との相関が見られない。活性化関数の他に精度をあげている要因があると考えられる。今回の実験では最適化関数について変更を行っていため、最適化関数に原因があるのではないかと考える。

以上より、エポック数による精度の向上はやや見られたが、活性化関数が識別にどのような効果をもたらしているのかは読み取ることができなかった。

%追加する時のtemplate
\begin{table}[htb]
  \begin{tabular}{|l|c|c|}
    \hline
    hogehoge & subjectID accuracy & fingerNum accuracy  \\ \hline
    hoge & hoge \% & hoge \% \\ \hline
    hoge & hoge \% & hoge \% \\ \hline
  \end{tabular}
\end{table}


\section{意図していた実験計画との違い}
%グループワークとして2ヶ月程度の時間が用意されていた。
%ガントチャート\ref{ganttchart}等、何かしら工夫して全体の計画を述べよう。
%これらの期間をどのように使おうとし、実際どうだったのかについて自己評価(振り返り)してみよう。
%大きなズレがある場合それは何故起きたのか、どうやればそのギャップを縮められそうか検討してみよう。
当初予定していた実験計画は以下の通りとなる。
\begin{itemize}
\item	6週目(11/17):テーマ決め、データセット探し、あと色々
\item	7週目(11/24):特徴ベクトルの抽出とかコード探して理解したりとか
\item	8週目(12/1)  :画像認識をコードに落とし込む
\item	9週目(12/8)  :実験開始?出来上がったコードを動かす段階?
\item	10週目(12/15):コードや特徴ベクトルの調整1
\item	11週目(12/22):
\item	12週目(1/5)  :
\item	13週目(1/12):改善実験?
\item	14週目(1/19):
\item	15週目(1/26):レポート・プレゼン資料作成
\item	期末テスト日(2/2):最終発表
\end{itemize}
上記の予定は11/17日に作成されたもの。これ以降に追加された締め切り予定に、
\begin{itemize}
\item	15週目(1/26):レポート初期版の提出日
\item	(2/16):Github公開
\end{itemize}
がある。

実際の進捗の経過をまとめると、下記の通りとなる。
\begin{itemize}
\item	6週目(11/17):テーマ決定(画像認証/fingerprint)・データセット探し、画像認証に対しての学習
\item	7週目(11/24):指紋の特徴量の分析、CNNの画像認証コードの検索
\item	8週目(12/1)  :参考するCNNコードの決定、コードの内容の分析・実行%私だけPython消してました
\item	9週目(12/8)  :コードの内容の分析・実行2nd
\item	10週目(12/15):コードの内容の分析3rd・amaneによる実行開始・実験目的の草案
\item	11週目(12/22):コードの内容の分析4th・実行2nd。またamaneで画像データの出力は難しいと判断。来年度に向けたの引継ぎ
\item	12週目(1/5)  :コードの内容の分析5th・epoch数を変えた大規模実験・Python/Keras環境の統一
\item	13週目(1/12):コードの内容の分析6th・batch\_sizeやLeakyReLUを変更した実験
\item	14週目(1/19):レポートの作成・amaneによる実験
\item	15週目(1/26):
\item	期末テスト日(2/2):
\end{itemize}

l'.').oO(だいぶふわふわした計画とは違うけど、まあ順調に進んでるのでは?)

%\usepackage{pgfgantt}    ←追加内容←追加しませんでした


\section{まとめ}
nothing else


\end{document}
